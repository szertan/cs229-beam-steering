\documentclass[20pt, custom, landscape]{tikzposter}
\geometry{paperwidth=72in, paperheight=36in}

% Packages
\usepackage{amsmath}
\usepackage{amssymb}
\usepackage{graphicx}
\usepackage{booktabs}
\usepackage{url}

% Stanford colors
\definecolor{StanfordRed}{RGB}{140, 21, 21}
\definecolor{StanfordWhite}{RGB}{255, 255, 255}

% Color theme - Stanford red headers with white text
\definecolorpalette{StanfordPalette}{
    \definecolor{colorOne}{RGB}{140, 21, 21}
    \definecolor{colorTwo}{RGB}{140, 21, 21}
    \definecolor{colorThree}{RGB}{255, 255, 255}
}

\usetheme{Default}
\usecolorstyle[colorPalette=StanfordPalette]{Britain}

% Block style - red header bar with white text
\defineblockstyle{StanfordBlock}{
    titlewidthscale=1, bodywidthscale=1, titleleft,
    titleoffsetx=0pt, titleoffsety=0pt, bodyoffsetx=0pt, bodyoffsety=0pt,
    bodyverticalshift=0pt, roundedcorners=0, linewidth=2pt,
    titleinnersep=8mm, bodyinnersep=10mm
}{
    \draw[color=StanfordRed, fill=StanfordWhite, line width=\blocklinewidth]
        (blockbody.south west) rectangle (blockbody.north east);
    \ifBlockHasTitle
        \draw[color=StanfordRed, fill=StanfordRed, line width=\blocklinewidth]
            (blocktitle.south west) rectangle (blocktitle.north east);
    \fi
}
\useblockstyle{StanfordBlock}

% White text on red title bars
\colorlet{blocktitlebgcolor}{StanfordRed}
\colorlet{blocktitlefgcolor}{white}

% Title formatting
\makeatletter
\settitle{
    \centering
    \vbox{
        \centering
        \color{StanfordRed}
        {\bfseries \Huge \@title \par}
        \vspace*{0.5em}
        {\Large \@author \par}
        \vspace*{0.3em}
        {\large \@institute}
    }
}
\makeatother

% Metadata
\title{Conditional Beam Steering in Plasma Photonic Crystals via Evolution Strategies}
\author{Selin Ertan, Computer Science}
\institute{Stanford University}

\begin{document}

\maketitle

\begin{columns}
\column{0.6}

% Header I - Introduction (compact)
\block{Introduction \& Physical System}{
    \textbf{Research Question:} Can gradient-free optimization match gradient-based methods for inverse EM design?
    
    \vspace{0.3em}
    \begin{minipage}{0.35\linewidth}
        \centering
        \includegraphics[width=0.9\linewidth]{figures/domain_schematic.pdf}
    \end{minipage}
    \hfill
    \begin{minipage}{0.6\linewidth}
        \textbf{Plasma rods:} $\varepsilon(\rho) = 1 - (\rho \cdot \omega_p^{\max}/\omega)^2$, \ $\varepsilon \in [-5.25, 1]$
        
        \textbf{Design space:} $\boldsymbol{\rho} \in [0,1]^{64}$ (64 rod densities)
        
        \textbf{Simulation:} FDFD solver, $400\times400$ grid, 6 GHz
    \end{minipage}
}

\column{0.4}

% Header II - Methods
\block{Methods}{
    \textbf{ES-Single:} Separate designs for each $\theta^* \in \{0^\circ, 90^\circ, 180^\circ\}$
    \vspace{-0.3em}
    \begin{equation*}
        R = P_{\theta^*} - 0.5 \sum_{\theta \neq \theta^*} P_\theta
    \end{equation*}
    
    \textbf{ES-Multi:} One design for all angles
    \vspace{-0.3em}
    \begin{equation*}
        R = \sum_\theta P_\theta - 0.5 \cdot \text{Var}(P_\theta)
    \end{equation*}
    
    \textbf{ES+NN:} Neural network $f_\phi: [\sin\theta, \cos\theta] \mapsto \boldsymbol{\rho}_{8\times8}$
}

% Header III - Algorithms
\block{ES Algorithm}{
    \textbf{1.} Sample $N=100$ perturbations $\boldsymbol{\epsilon}_i \sim \mathcal{N}(0,I)$
    
    \textbf{2.} Evaluate $R_i = R(\boldsymbol{\theta} + \sigma \boldsymbol{\epsilon}_i)$ in parallel
    
    \textbf{3.} Estimate gradient: $\hat{\mathbf{g}} = \frac{1}{N\sigma} \sum w_i \boldsymbol{\epsilon}_i$
    
    \textbf{4.} Update with Adam ($\eta=0.02$, $\sigma_0=0.3$)
    
    \vspace{0.5em}
    \textbf{ES+NN:} Same procedure, but $\boldsymbol{\theta}$ are NN weights (20,928 params). Network: 4-layer MLP, input $[\sin\theta, \cos\theta]$, output $\boldsymbol{\rho}_{8\times8}$
}

\end{columns}

\begin{columns}
\column{0.75}

% Header IV - Results
\block{Results}{
    \begin{minipage}{0.48\linewidth}
        \centering
        \includegraphics[width=0.85\linewidth]{figures/field_patterns.pdf}
        
        \small ES-Single: $<$0.005\% crosstalk
    \end{minipage}
    \hfill
    \begin{minipage}{0.48\linewidth}
        \centering
        \includegraphics[width=0.85\linewidth]{figures/nn_mode_collapse.pdf}
        
        \small ES+NN: $P_0 = 9.8\times10^5$, $P_{90} = 4.7\times10^5$, $P_{180} = 5.5$
    \end{minipage}
}

\column{0.25}

% Header V - Conclusions
\block{Conclusions}{
    $\bullet$ \textbf{ES-Single} achieves $<$0.005\% crosstalk
    
    $\bullet$ \textbf{ES-Multi} confirms no universal design exists
    
    $\bullet$ \textbf{ES+NN} reveals: $0^\circ/90^\circ$ compatible, $180^\circ$ incompatible
    
    $\bullet$ Gradient-free approach is \textbf{viable}
    
    \vspace{0.5em}
    \small
    \textbf{Code:} \url{github.com/szertan/cs229-beam-steering}
}

\end{columns}

\end{document}
